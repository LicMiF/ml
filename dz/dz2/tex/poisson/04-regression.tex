\item \subquestionpoints{36} \textbf{Програмерски задатак}

Размотрити интернет страницу која жели да предвиди дневни саобраћај. Власници интернет странице су сакупили скуп података саобраћаја ка својој интернет страници из прошлости и уз то још нека својства за која су мислили да могу бити корисна у предвиђању дневног броја посетилаца. Скуп података је подељен на тренинг скуп и валидациони скуп и шаблон изворног кода је дат у следећим датотекама:
\begin{center}
\begin{itemize}
\item 	\url{src/poisson/{train,valid}.csv}
\item   \url{src/poisson/poisson.py}
\end{itemize}
\end{center}
Биће применења Поасонова регресија за моделовање дневног броја посетилаца. Треба напоменути да сама примена Поасонове регресије унапред претпоставља да подаци прате Поасонову расподелу чији је природни параметар линеарна комбинација улазних атрибута (то јест, $\eta = \theta^T x$). У \texttt{src/poisson/poisson.py}, имплементирати Поасонову регресију за дати скуп података и користити \emph{потпун (нестохастички) градијентни успон} за максимизацију логаритамске веродостојности параметра $\theta$. Као критеријум за заустављање проверити да ли промена параметара има норму која је мања од вредности $10^{-5}$.

Користећи истрениран модел предвидети очекиван број на \textbf{валидационом скупу} и направити дијаграм расејања између тачног броја и предвиђеног броја посета (на валидационом скупу). На дијаграму расејања нека на апциси буду тачне вредности за број посета, а а на ординати одговарајућа предвиђања очекиваног броја посета. Приметити да су тачне вредности целобројне, док очекиване вредности су у општем случају реалне.
