\item\subquestionpoints{14} {\bf Нормална расподела са различитим варијансама}

Претпоставити да је дат скуп података $\{(x^{(i)},y^{(i)});i=1,\ldots,m\}$ са $m$ независних примера у којима $y^{(i)}$ прати условну расподелу вероватноће са различитим нивоима варијансе $(\sigma^{(i)})^2$. Конкретно, претпоставка се може изразити и расподелом вероватноће
\begin{equation*}
 p(y^{(i)}|x^{(i)};\theta)=\dfrac{1}{\sqrt{2\pi}\sigma^{(i)}}\exp\left(-\dfrac{\left(y^{(i)}-\theta^Tx^{(i)}\right)^2}{2\left(\sigma^{(i)}\right)^2}\right)\quad,
\end{equation*}
то јест, свако $y^{(i)}$ се извлачи из нормалне, односно Гаусове, расподеле са средњом вредношћу $\theta^Tx^{(i)}$ и варијансом $(\sigma^{(i)})^2$ где су $\sigma^{(i)}$ унапред познате константе. Показати да се проналажење $\theta$ методом највеће веродостојности своди на решавање задатка локално-тежинске линеарне регресије. Јасно изразити тежинске коефицијенте $w^{(i)}$ у функцији стандардне дефијације $\sigma^{(i)}$.

