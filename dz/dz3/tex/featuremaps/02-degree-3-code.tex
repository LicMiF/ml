\item\subquestionpoints{18} {\bf Програмерски задатак: регресија полиномом трећег степена}

За овај подзадатак биће коришћен скуп података дат у следећим датотекама:
%
\begin{center}
\url{src/featuremaps/{train,valid,test}.csv}
\end{center}
%
Свака датотека садржи две колоне: $x$ и $y$. У терминологији описаној у уводу, $x$ је атрибут (у овом случају једнодимензионални), а $y$ је излазна ознака.

Користећи формулацију из претходног подзадатка имплементирати линеарну регресију са \textbf{нормалном једначином} користећи преслигавање својстава полиномом трећег степена. Користити шаблон који је обезбеђен у \texttt{src/featuremaps/featuremap.py} да би се имплементирао алгоритам.

Направити дијаграм расејања тренинг података и нацртати научену хипотезу као глатку криву над истим. Укључити дијаграм у извештај као решење овог задатка.

\emph{Напомена:} Претпоставити да је $\widehat{X}$ матрица података трансформисаног скупа података. Понекад се може срести нерегуларна, то јест сингуларна, матрица $\widehat{X}^T\widehat{X}$. Да би се добило нумерички стабилно решење увек треба користити \texttt{np.linalg.solve} како би се параметри добили директно, уместо да се експлицитно израчунава инверзија, а затим множи са $\widehat{X}^Ty$.

