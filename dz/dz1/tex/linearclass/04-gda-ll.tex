\item \subquestionpoints{15} За задати скуп података, тврди се да су на основу методе највеће веродостојности (МНВ) параметри дати као
  \begin{eqnarray*}
    \phi &=& \frac{1}{\nexp} \sum_{i=1}^\nexp 1\{y^{(i)} = 1\} \\
\mu_{0} &=& \frac{\sum_{i=1}^\nexp 1\{y^{(i)} = {0}\} x^{(i)}}{\sum_{i=1}^\nexp
1\{y^{(i)} = {0}\}} \\
\mu_1 &=& \frac{\sum_{i=1}^\nexp 1\{y^{(i)} = 1\} x^{(i)}}{\sum_{i=1}^\nexp 1\{y^{(i)}
= 1\}} \\
\Sigma &=& \frac{1}{\nexp} \sum_{i=1}^\nexp (x^{(i)} - \mu_{y^{(i)}}) (x^{(i)} -
\mu_{y^{(i)}})^T
  \end{eqnarray*}
  Логаритамска функција веродостојности података је 
  \begin{eqnarray*}
\ell(\phi, \mu_{0}, \mu_1, \Sigma) &=& \log \prod_{i=1}^\nexp p(x^{(i)} , y^{(i)};
\phi, \mu_{0}, \mu_1, \Sigma) \\
&=& \log \prod_{i=1}^\nexp p(x^{(i)} | y^{(i)}; \mu_{0}, \mu_1, \Sigma) p(y^{(i)};
\phi).
  \end{eqnarray*}
Максимизацијом $\ell$ по четири параметра, доказати да су процене $\phi$, $\mu_{0}, \mu_1$, и
$\Sigma$ методом највеће веродостојности заиста онакве као у горњим једнакостима. (Може се претпоставити да постоји бар један позитиван и макар један негативан пример тако да су имениоци у дефиницијама за $\mu_{0}$ и $\mu_1$ различити од нуле.)

