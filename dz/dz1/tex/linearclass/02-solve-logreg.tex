\item \subquestionpoints{10} \textbf{Програмерски задатак.}
Пратити упутства дата у \texttt{src/linearclass/logreg.py} да се истренира класификатор заснован на логистичкој регресији користећи се Њутновом методом. Почевши од $\theta = \vec{0}$, извршавати Њутнову методу све док померај по $\theta$ не постане мали: Конкретно, тренирати до прве итерације $k$ за коју важи $\|\theta_{k} - \theta_{k-1}\|_1 < \epsilon$, где је $\epsilon = 1\times 10^{-5}$. Обавезно уписати вероватноће предвиђања на валидационом скупу у датотеку која је дата у изворном коду.

Укључити график \textbf{валидационих података} са $x_1$ на хоризонталној оси и $x_2$ на вертикалној оси. За представљање две класе користити различите маркере (симболе) за примере $x^{(i)}$ за које је $y^{(i)} = 0$ у односу на оне за које је $y^{(i)} = 1$. На истом графику исцртати границу одлуке коју проналази логистичка регресија (тј. праву која одговара $p(y|x) = 0.5$).

