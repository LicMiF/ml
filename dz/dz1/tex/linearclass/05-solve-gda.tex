\item \subquestionpoints{10} \textbf{Програмерски задатак.}
У датотеци \texttt{src/linearclass/gda.py} допунити изворни код тако да израчунава $\phi$, $\mu_{0}$, $\mu_{1}$, и $\Sigma$, затим искористити ове параметре да се добије $\theta$, и коначно употребити тако добијени ГДА модел за предвиђања на валидационом скупу података. Обавезно уписати вероватноће предвиђања на валидационом скупу у датотеку коjа jе дата у изворном коду.

Укључити график \textbf{валидационих података} са $x_1$ на хоризонталноj оси и $x_2$ на вертикалноj оси. За представљање две класе користити различите маркере (симболе)
за примере $x^{(i)}$ за коjе jе $y^{(i)} = 0$ у односу на оне за коjе jе $y^{(i)} = 1$. На истом графику исцртати границу одлуке коjу проналази ГДА (тj. праву коjа одговара $p(y|x) = 0.5$).

