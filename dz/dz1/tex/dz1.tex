

\def\qtr{јесен 2021.}
\def\due{понедељак, 8. новембар у 23:59}
\def\psetnum{1}

%%% Change the following flag to toggle between questions or solutions
\ifdefined\solutions {} \else \def\solutions{1} \fi


\documentclass{article}

\usepackage[utf8]{inputenc}
%\usepackage[T1,T2A]{fontenc}
\usepackage[T2A]{fontenc}
\usepackage[serbian]{babel}

\usepackage{graphicx}

\newcommand{\di}{{d}}
\newcommand{\nexp}{{n}}
\newcommand{\nf}{{p}}
\newcommand{\vcd}{{\textbf{D}}}

\usepackage{nccmath}
\usepackage{mathtools}
\usepackage{graphicx,caption}
\usepackage{enumitem}
\usepackage{epstopdf,subcaption}
\usepackage{psfrag}
\usepackage{amsmath,amssymb,epsf}
\usepackage{verbatim}
\usepackage[hyphens]{url}
\usepackage{color}
\usepackage{bbm}
\usepackage{listings}
\usepackage{setspace}
\usepackage{float}
\usepackage{natbib}
\definecolor{Code}{rgb}{0,0,0}
\definecolor{Decorators}{rgb}{0.5,0.5,0.5}
\definecolor{Numbers}{rgb}{0.5,0,0}
\definecolor{MatchingBrackets}{rgb}{0.25,0.5,0.5}
\definecolor{Keywords}{rgb}{0,0,1}
\definecolor{self}{rgb}{0,0,0}
\definecolor{Strings}{rgb}{0,0.63,0}
\definecolor{Comments}{rgb}{0,0.63,1}
\definecolor{Backquotes}{rgb}{0,0,0}
\definecolor{Classname}{rgb}{0,0,0}
\definecolor{FunctionName}{rgb}{0,0,0}
\definecolor{Operators}{rgb}{0,0,0}
\definecolor{Background}{rgb}{0.98,0.98,0.98}
\lstdefinelanguage{Python}{
numbers=left,
numberstyle=\footnotesize,
numbersep=1em,
xleftmargin=1em,
framextopmargin=2em,
framexbottommargin=2em,
showspaces=false,
showtabs=false,
showstringspaces=false,
frame=l,
tabsize=4,
% Basic
basicstyle=\ttfamily\footnotesize\setstretch{1},
backgroundcolor=\color{Background},
% Comments
commentstyle=\color{Comments}\slshape,
% Strings
stringstyle=\color{Strings},
morecomment=[s][\color{Strings}]{"""}{"""},
morecomment=[s][\color{Strings}]{'''}{'''},
% keywords
morekeywords={import,from,class,def,for,while,if,is,in,elif,else,not,and,or
,print,break,continue,return,True,False,None,access,as,,del,except,exec
,finally,global,import,lambda,pass,print,raise,try,assert},
keywordstyle={\color{Keywords}\bfseries},
% additional keywords
morekeywords={[2]@invariant},
keywordstyle={[2]\color{Decorators}\slshape},
emph={self},
emphstyle={\color{self}\slshape},
%
}


\pagestyle{empty} \addtolength{\textwidth}{1.0in}
\addtolength{\textheight}{0.5in}
\addtolength{\oddsidemargin}{-0.5in}
\addtolength{\evensidemargin}{-0.5in}
\newcommand{\ruleskip}{\bigskip\hrule\bigskip}
\newcommand{\nodify}[1]{{\sc #1}}
\newcommand{\points}[1]{{\textbf{[#1 поена]}}}
\newcommand{\subquestionpoints}[1]{{[#1 поена]}}
\newenvironment{answer}{{\bf Одговор:} \sf \begingroup\color{red}}{\endgroup}%

\newcommand{\bitem}{\begin{list}{$\bullet$}%
{\setlength{\itemsep}{0pt}\setlength{\topsep}{0pt}%
\setlength{\rightmargin}{0pt}}}
\newcommand{\eitem}{\end{list}}

\setlength{\parindent}{0pt} \setlength{\parskip}{0.5ex}
\setlength{\unitlength}{1cm}

\renewcommand{\Re}{{\mathbb R}}
\newcommand{\R}{\mathbb{R}}
\newcommand{\what}[1]{\widehat{#1}}

\renewcommand{\comment}[1]{}
\newcommand{\mc}[1]{\mathcal{#1}}
\newcommand{\half}{\frac{1}{2}}

\def\KL{D_{KL}}
\def\xsi{x^{(i)}}
\def\ysi{y^{(i)}}
\def\zsi{z^{(i)}}
\def\E{\mathbb{E}}
\def\calN{\mathcal{N}}
\def\calD{\mathcal{D}}

\usepackage{tikz}
\usepackage{bbding}
\usepackage{pifont}
\usepackage{wasysym}
\usepackage{amssymb}
\usepackage{booktabs}
\usepackage{verbatim}



\def\shownotes{0}  %set 1 to show author notes
\ifnum\shownotes=1
\newcommand{\authnote}[2]{$\ll$\textsf{\footnotesize #1 notes: #2}$\gg$}
\else
\newcommand{\authnote}[2]{}
\fi

\newcommand{\tnote}[1]{{\color{blue}\authnote{Влада}{#1}}}
\newcommand{\fk}[1]{{\color{purple}{[FK:#1]}}}
\newcommand{\notes}[1]{{\color{blue} Note:} \textit{#1} \newline}


\begin{document}

\pagestyle{myheadings} \markboth{}{Основи машинског учења --- домаћи задатак \textnumero\psetnum}

\ifnum\solutions=1{
{\huge\noindent Основи машинског учења, \qtr\\
домаћи задатак \textnumero\psetnum\;решења}\\\\
ИМЕ И ПРЕЗИМЕ (\texttt{БРОЈ ИНДЕКСА})
} \else {\huge
\noindent Основи машинског учења, \qtr\\
домаћи задатак \textnumero\psetnum
} \fi

\ruleskip

{\bf Рок: {\due } на Moodle-у.}

\medskip

\item \points{90} {\bf Линеарни класификатори (логистичка регресија и ГДА)}

У овом задатку, биће покривена два линеарна класификатора која су до сада обрађена на предавањима. Први, дискриминативни линеарни класификатор: логистичка регресија. Други, генеративни линеарни класификатор: Гаусова дискриминантна анализа (ГДА). Оба алгоритма проналазе линеарну границу одлуке која раздваја податке на две класе, али уз различите претпоставке. Циљ овог домаћег задатка јесте да се стекне дубље разумевање о сличностима и разликама (као и о предностима и манама) ова два алгоритма.

У склопу овог задатка, биће размотрена два скупа података, уз шаблоне изворних кодова који су дати у следећим датотекама:
\begin{center}
\begin{itemize} %[label=\roman*.]
\item \url{src/linearclass/ds1_{train,valid}.csv}
\item \url{src/linearclass/ds2_{train,valid}.csv}
\item \url{src/linearclass/logreg.py}
\item \url{src/linearclass/gda.py}
\end{itemize}
\end{center}
Свака датотека садржи $\nexp$ примера, један пример $(x^{(i)}, y^{(i)})$ по реду. Нарочито, $i$-ти ред садржи колоне $x^{(i)}_1\in\Re$,
$x^{(i)}_2\in\Re$, и $y^{(i)}\in\{0, 1\}$. У подзадацима који следе, биће испитано коришћење логистичке регресије и Гаусове дискриминантне анализе (ГДА) како би се извршила двојна (бинарна) класификација на ова два скупа података.

\begin{enumerate}
	\item \subquestionpoints{20}

На предавањима је приказана функција губитака за логистичку регресију:
\begin{equation*}
	J(\theta)
	= -\frac{1}{\nexp} \sum_{i=1}^\nexp \left(y^{(i)}\log(h_{\theta}(x^{(i)}))
		+  (1 - y^{(i)})\log(1 - h_{\theta}(x^{(i)}))\right),
\end{equation*}
где је $y^{(i)} \in \{0, 1\}$, $h_\theta(x) = g(\theta^T x)$ и $g(z) = 1 / (1 + e^{-z})$.

Пронаћи Хесијан $H$ ове функције и показати да за произвољни вектор $z$ важи
%
\begin{equation*}
    z^T H z \ge 0.
\end{equation*}
%
{\bf Смерница:} Може се најпре показати да је $\sum_i\sum_j z_i x_i x_j z_j = (x^Tz)^2 \geq 0$. Подсетити се такође да је $g'(z) = g(z)(1-g(z))$.

{\bf Напомена:} Ово је један од уобичајених начина да се покаже да је матрица $H$ позитивно семидефинитна, што се означава са ``$H \succeq 0$.'' Ово даље имплицира да је $J$ конвексна, односно да нема других локалних минимума изузев глобалног. Није неопходно користити горњу смерницу како би се показало да је $H \succeq 0$ већ било коју.


        \ifnum\solutions=1 {
            \begin{answer}
\end{answer}

        } \fi

	\item \subquestionpoints{10} \textbf{Програмерски задатак.}
Пратити упутства дата у \texttt{src/linearclass/logreg.py} да се истренира класификатор заснован на логистичкој регресији користећи се Њутновом методом. Почевши од $\theta = \vec{0}$, извршавати Њутнову методу све док померај по $\theta$ не постане мали: Конкретно, тренирати до прве итерације $k$ за коју важи $\|\theta_{k} - \theta_{k-1}\|_1 < \epsilon$, где је $\epsilon = 1\times 10^{-5}$. Обавезно уписати вероватноће предвиђања на валидационом скупу у датотеку која је дата у изворном коду.

Укључити график \textbf{валидационих података} са $x_1$ на хоризонталној оси и $x_2$ на вертикалној оси. За представљање две класе користити различите маркере (симболе) за примере $x^{(i)}$ за које је $y^{(i)} = 0$ у односу на оне за које је $y^{(i)} = 1$. На истом графику исцртати границу одлуке коју проналази логистичка регресија (тј. праву која одговара $p(y|x) = 0.5$).


        \ifnum\solutions=1 {
            \begin{answer}
\end{answer}

        } \fi


	\item \subquestionpoints{10}
Подсетити се да је у ГДА заједничка расподела $(x, y)$ описана следећим једначинама:
%
\begin{eqnarray*}
	p(y) &=& \begin{cases}
	\phi & \mbox{if~} y = 1 \\
	1 - \phi & \mbox{if~} y = 0 \end{cases} \\
	p(x | y=0) &=& \frac{1}{(2\pi)^{\di/2} |\Sigma|^{1/2}}
		\exp\left(-\frac{1}{2}(x-\mu_{0})^T \Sigma^{-1} (x-\mu_{0})\right) \\
	p(x | y=1) &=& \frac{1}{(2\pi)^{\di/2} |\Sigma|^{1/2}}
		\exp\left(-\frac{1}{2}(x-\mu_1)^T \Sigma^{-1} (x-\mu_1) \right),
\end{eqnarray*}
%
где су $\phi$, $\mu_0$, $\mu_1$, и $\Sigma$ параметри модела.

Претпоставимо да су $\phi$, $\mu_0$, $\mu_1$, и $\Sigma$ већ одређени и да је даље неопходно предвидети $y$ за нову задату тачку $x$. Како би се доказало да ГДА као резултат даје класификатор са линеарном границом одлуке, показати да се апостериорна вероватноћа може написати као
%
\begin{equation*}
	p(y = 1\mid x; \phi, \mu_0, \mu_1, \Sigma)
	= \frac{1}{1 + \exp(-(\theta^T x + \theta_0))},
\end{equation*}
%
где су $\theta\in\Re^\di$ и $\theta_{0}\in\Re$ одговарајуће функције параметара $\phi$, $\Sigma$, $\mu_0$, и $\mu_1$.


        \ifnum\solutions=1 {
            \begin{answer}
\end{answer}

        }\fi

	\item \subquestionpoints{15} За задати скуп података, тврди се да су на основу методе највеће веродостојности (МНВ) параметри дати као
  \begin{eqnarray*}
    \phi &=& \frac{1}{\nexp} \sum_{i=1}^\nexp 1\{y^{(i)} = 1\} \\
\mu_{0} &=& \frac{\sum_{i=1}^\nexp 1\{y^{(i)} = {0}\} x^{(i)}}{\sum_{i=1}^\nexp
1\{y^{(i)} = {0}\}} \\
\mu_1 &=& \frac{\sum_{i=1}^\nexp 1\{y^{(i)} = 1\} x^{(i)}}{\sum_{i=1}^\nexp 1\{y^{(i)}
= 1\}} \\
\Sigma &=& \frac{1}{\nexp} \sum_{i=1}^\nexp (x^{(i)} - \mu_{y^{(i)}}) (x^{(i)} -
\mu_{y^{(i)}})^T
  \end{eqnarray*}
  Логаритамска функција веродостојности података је 
  \begin{eqnarray*}
\ell(\phi, \mu_{0}, \mu_1, \Sigma) &=& \log \prod_{i=1}^\nexp p(x^{(i)} , y^{(i)};
\phi, \mu_{0}, \mu_1, \Sigma) \\
&=& \log \prod_{i=1}^\nexp p(x^{(i)} | y^{(i)}; \mu_{0}, \mu_1, \Sigma) p(y^{(i)};
\phi).
  \end{eqnarray*}
Максимизацијом $\ell$ по четири параметра, доказати да су процене $\phi$, $\mu_{0}, \mu_1$, и
$\Sigma$ методом највеће веродостојности заиста онакве као у горњим једнакостима. (Може се претпоставити да постоји бар један позитиван и макар један негативан пример тако да су имениоци у дефиницијама за $\mu_{0}$ и $\mu_1$ различити од нуле.)


        \ifnum\solutions=1 {
            \begin{answer}
\end{answer}

        } \fi

	\item \subquestionpoints{10} \textbf{Програмерски задатак.}
У датотеци \texttt{src/linearclass/gda.py} допунити изворни код тако да израчунава $\phi$, $\mu_{0}$, $\mu_{1}$, и $\Sigma$, затим искористити ове параметре да се добије $\theta$, и коначно употребити тако добијени ГДА модел за предвиђања на валидационом скупу података. Обавезно уписати вероватноће предвиђања на валидационом скупу у датотеку коjа jе дата у изворном коду.

Укључити график \textbf{валидационих података} са $x_1$ на хоризонталноj оси и $x_2$ на вертикалноj оси. За представљање две класе користити различите маркере (симболе)
за примере $x^{(i)}$ за коjе jе $y^{(i)} = 0$ у односу на оне за коjе jе $y^{(i)} = 1$. На истом графику исцртати границу одлуке коjу проналази ГДА (тj. праву коjа одговара $p(y|x) = 0.5$).


        \ifnum\solutions=1 {
            \begin{answer}
\end{answer}

        } \fi

	\item \subquestionpoints{5}
За први скуп података (\texttt{ds1\_valid}) упоредити графике добијене из логистичке регресије и ГДА из претходних подзадатака и укратко у пар редова прокоментарисати запажања.


        \ifnum\solutions=1 {
            \begin{answer}
\end{answer}

        } \fi

	\item \subquestionpoints{10}
Поновити програмерске подзадатке за други скуп података. Направити сличне графике на \textbf{валидационом скупу} и укључити их у одговор.

На ком од два скупа података ГДА ради лошије од логистичке регресије? Шта може бити узрок томе?


        \ifnum\solutions=1{
            \begin{answer}
\end{answer}

        }\fi

	\item \subquestionpoints{10} За скуп података на ком ГДА ради лошије, испитати да ли је могуће пронаћи трансформацију улазних података $x^{(i)}$ такву да ГДА ради знатно боље? Која би то трансформација могла бити?


        \ifnum\solutions=1{
            \begin{answer}
\end{answer}

        }\fi

\end{enumerate}


\begin{enumerate}[wide, labelwidth=!, labelindent=0pt]

\clearpage
\item \points{90} {\bf Линеарни класификатори (логистичка регресија и ГДА)}

У овом задатку, биће покривена два линеарна класификатора која су до сада обрађена на предавањима. Први, дискриминативни линеарни класификатор: логистичка регресија. Други, генеративни линеарни класификатор: Гаусова дискриминантна анализа (ГДА). Оба алгоритма проналазе линеарну границу одлуке која раздваја податке на две класе, али уз различите претпоставке. Циљ овог домаћег задатка јесте да се стекне дубље разумевање о сличностима и разликама (као и о предностима и манама) ова два алгоритма.

У склопу овог задатка, биће размотрена два скупа података, уз шаблоне изворних кодова који су дати у следећим датотекама:
\begin{center}
\begin{itemize} %[label=\roman*.]
\item \url{src/linearclass/ds1_{train,valid}.csv}
\item \url{src/linearclass/ds2_{train,valid}.csv}
\item \url{src/linearclass/logreg.py}
\item \url{src/linearclass/gda.py}
\end{itemize}
\end{center}
Свака датотека садржи $\nexp$ примера, један пример $(x^{(i)}, y^{(i)})$ по реду. Нарочито, $i$-ти ред садржи колоне $x^{(i)}_1\in\Re$,
$x^{(i)}_2\in\Re$, и $y^{(i)}\in\{0, 1\}$. У подзадацима који следе, биће испитано коришћење логистичке регресије и Гаусове дискриминантне анализе (ГДА) како би се извршила двојна (бинарна) класификација на ова два скупа података.

\begin{enumerate}
	\item \subquestionpoints{20}

На предавањима је приказана функција губитака за логистичку регресију:
\begin{equation*}
	J(\theta)
	= -\frac{1}{\nexp} \sum_{i=1}^\nexp \left(y^{(i)}\log(h_{\theta}(x^{(i)}))
		+  (1 - y^{(i)})\log(1 - h_{\theta}(x^{(i)}))\right),
\end{equation*}
где је $y^{(i)} \in \{0, 1\}$, $h_\theta(x) = g(\theta^T x)$ и $g(z) = 1 / (1 + e^{-z})$.

Пронаћи Хесијан $H$ ове функције и показати да за произвољни вектор $z$ важи
%
\begin{equation*}
    z^T H z \ge 0.
\end{equation*}
%
{\bf Смерница:} Може се најпре показати да је $\sum_i\sum_j z_i x_i x_j z_j = (x^Tz)^2 \geq 0$. Подсетити се такође да је $g'(z) = g(z)(1-g(z))$.

{\bf Напомена:} Ово је један од уобичајених начина да се покаже да је матрица $H$ позитивно семидефинитна, што се означава са ``$H \succeq 0$.'' Ово даље имплицира да је $J$ конвексна, односно да нема других локалних минимума изузев глобалног. Није неопходно користити горњу смерницу како би се показало да је $H \succeq 0$ већ било коју.


        \ifnum\solutions=1 {
            \begin{answer}
\end{answer}

        } \fi

	\item \subquestionpoints{10} \textbf{Програмерски задатак.}
Пратити упутства дата у \texttt{src/linearclass/logreg.py} да се истренира класификатор заснован на логистичкој регресији користећи се Њутновом методом. Почевши од $\theta = \vec{0}$, извршавати Њутнову методу све док померај по $\theta$ не постане мали: Конкретно, тренирати до прве итерације $k$ за коју важи $\|\theta_{k} - \theta_{k-1}\|_1 < \epsilon$, где је $\epsilon = 1\times 10^{-5}$. Обавезно уписати вероватноће предвиђања на валидационом скупу у датотеку која је дата у изворном коду.

Укључити график \textbf{валидационих података} са $x_1$ на хоризонталној оси и $x_2$ на вертикалној оси. За представљање две класе користити различите маркере (симболе) за примере $x^{(i)}$ за које је $y^{(i)} = 0$ у односу на оне за које је $y^{(i)} = 1$. На истом графику исцртати границу одлуке коју проналази логистичка регресија (тј. праву која одговара $p(y|x) = 0.5$).


        \ifnum\solutions=1 {
            \begin{answer}
\end{answer}

        } \fi


	\item \subquestionpoints{10}
Подсетити се да је у ГДА заједничка расподела $(x, y)$ описана следећим једначинама:
%
\begin{eqnarray*}
	p(y) &=& \begin{cases}
	\phi & \mbox{if~} y = 1 \\
	1 - \phi & \mbox{if~} y = 0 \end{cases} \\
	p(x | y=0) &=& \frac{1}{(2\pi)^{\di/2} |\Sigma|^{1/2}}
		\exp\left(-\frac{1}{2}(x-\mu_{0})^T \Sigma^{-1} (x-\mu_{0})\right) \\
	p(x | y=1) &=& \frac{1}{(2\pi)^{\di/2} |\Sigma|^{1/2}}
		\exp\left(-\frac{1}{2}(x-\mu_1)^T \Sigma^{-1} (x-\mu_1) \right),
\end{eqnarray*}
%
где су $\phi$, $\mu_0$, $\mu_1$, и $\Sigma$ параметри модела.

Претпоставимо да су $\phi$, $\mu_0$, $\mu_1$, и $\Sigma$ већ одређени и да је даље неопходно предвидети $y$ за нову задату тачку $x$. Како би се доказало да ГДА као резултат даје класификатор са линеарном границом одлуке, показати да се апостериорна вероватноћа може написати као
%
\begin{equation*}
	p(y = 1\mid x; \phi, \mu_0, \mu_1, \Sigma)
	= \frac{1}{1 + \exp(-(\theta^T x + \theta_0))},
\end{equation*}
%
где су $\theta\in\Re^\di$ и $\theta_{0}\in\Re$ одговарајуће функције параметара $\phi$, $\Sigma$, $\mu_0$, и $\mu_1$.


        \ifnum\solutions=1 {
            \begin{answer}
\end{answer}

        }\fi

	\item \subquestionpoints{15} За задати скуп података, тврди се да су на основу методе највеће веродостојности (МНВ) параметри дати као
  \begin{eqnarray*}
    \phi &=& \frac{1}{\nexp} \sum_{i=1}^\nexp 1\{y^{(i)} = 1\} \\
\mu_{0} &=& \frac{\sum_{i=1}^\nexp 1\{y^{(i)} = {0}\} x^{(i)}}{\sum_{i=1}^\nexp
1\{y^{(i)} = {0}\}} \\
\mu_1 &=& \frac{\sum_{i=1}^\nexp 1\{y^{(i)} = 1\} x^{(i)}}{\sum_{i=1}^\nexp 1\{y^{(i)}
= 1\}} \\
\Sigma &=& \frac{1}{\nexp} \sum_{i=1}^\nexp (x^{(i)} - \mu_{y^{(i)}}) (x^{(i)} -
\mu_{y^{(i)}})^T
  \end{eqnarray*}
  Логаритамска функција веродостојности података је 
  \begin{eqnarray*}
\ell(\phi, \mu_{0}, \mu_1, \Sigma) &=& \log \prod_{i=1}^\nexp p(x^{(i)} , y^{(i)};
\phi, \mu_{0}, \mu_1, \Sigma) \\
&=& \log \prod_{i=1}^\nexp p(x^{(i)} | y^{(i)}; \mu_{0}, \mu_1, \Sigma) p(y^{(i)};
\phi).
  \end{eqnarray*}
Максимизацијом $\ell$ по четири параметра, доказати да су процене $\phi$, $\mu_{0}, \mu_1$, и
$\Sigma$ методом највеће веродостојности заиста онакве као у горњим једнакостима. (Може се претпоставити да постоји бар један позитиван и макар један негативан пример тако да су имениоци у дефиницијама за $\mu_{0}$ и $\mu_1$ различити од нуле.)


        \ifnum\solutions=1 {
            \begin{answer}
\end{answer}

        } \fi

	\item \subquestionpoints{10} \textbf{Програмерски задатак.}
У датотеци \texttt{src/linearclass/gda.py} допунити изворни код тако да израчунава $\phi$, $\mu_{0}$, $\mu_{1}$, и $\Sigma$, затим искористити ове параметре да се добије $\theta$, и коначно употребити тако добијени ГДА модел за предвиђања на валидационом скупу података. Обавезно уписати вероватноће предвиђања на валидационом скупу у датотеку коjа jе дата у изворном коду.

Укључити график \textbf{валидационих података} са $x_1$ на хоризонталноj оси и $x_2$ на вертикалноj оси. За представљање две класе користити различите маркере (симболе)
за примере $x^{(i)}$ за коjе jе $y^{(i)} = 0$ у односу на оне за коjе jе $y^{(i)} = 1$. На истом графику исцртати границу одлуке коjу проналази ГДА (тj. праву коjа одговара $p(y|x) = 0.5$).


        \ifnum\solutions=1 {
            \begin{answer}
\end{answer}

        } \fi

	\item \subquestionpoints{5}
За први скуп података (\texttt{ds1\_valid}) упоредити графике добијене из логистичке регресије и ГДА из претходних подзадатака и укратко у пар редова прокоментарисати запажања.


        \ifnum\solutions=1 {
            \begin{answer}
\end{answer}

        } \fi

	\item \subquestionpoints{10}
Поновити програмерске подзадатке за други скуп података. Направити сличне графике на \textbf{валидационом скупу} и укључити их у одговор.

На ком од два скупа података ГДА ради лошије од логистичке регресије? Шта може бити узрок томе?


        \ifnum\solutions=1{
            \begin{answer}
\end{answer}

        }\fi

	\item \subquestionpoints{10} За скуп података на ком ГДА ради лошије, испитати да ли је могуће пронаћи трансформацију улазних података $x^{(i)}$ такву да ГДА ради знатно боље? Која би то трансформација могла бити?


        \ifnum\solutions=1{
            \begin{answer}
\end{answer}

        }\fi

\end{enumerate}


\end{enumerate}

\end{document}
